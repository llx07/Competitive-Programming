\documentclass{article}
\usepackage[UTF8]{ctex}
\usepackage{fancyhdr}  % 页眉页脚
\usepackage{minted}    % 代码高亮
\usepackage[colorlinks,linkcolor=black]{hyperref}  % 目录可跳转
\usepackage[table,xcdraw]{xcolor} % 表格背景颜色
\usepackage{amsmath} % operatorname 用
\usepackage{float}
\usepackage{geometry} % 调整页边距

% 调整页边距
\geometry{a4paper,scale=0.8}


% \setlength{\headheight}{5pt}

% 定义页眉页脚
\pagestyle{fancy}
\fancyhf{}
\fancyhead[C]{ACM参考资料}
\lfoot{}
\cfoot{}
\rfoot{\thepage}
% \setlength{\headheight}{5pt}

\author{Lexiao Lin}   
\title{ACM参考资料}

\begin{document}
\maketitle % 封面
\newpage % 换页
\tableofcontents % 目录
\newpage

\section{基础知识}
\subsection{常用数学函数值}


\begin{table}[H]
\centering
\begin{tabular}{|l|l|l|l|l|}
\hline
\rowcolor[HTML]{C0C0C0} 
$n$ & $\log_{10} n$ & $n!$ & $C(n,n/2)$ & $\operatorname{lcm}(1...n)$ \\
2          & 0.301030   & 2              & 2              & 2              \\\hline
3          & 0.477121   & 6              & 3              & 6              \\\hline
4          & 0.602060   & 24             & 6              & 12             \\\hline
5          & 0.698970   & 120            & 10             & 60             \\\hline
6          & 0.778151   & 720            & 20             & 60             \\\hline
7          & 0.845098   & 5040           & 35             & 420            \\\hline
8          & 0.903090   & 40320          & 70             & 840            \\\hline
9          & 0.954243   & 362880         & 126            & 2520           \\\hline
10         & 1.000000   & 3628800        & 252            & 2520           \\\hline
11         & 1.041393   & 39916800       & 462            & 27720          \\\hline
12         & 1.079181   & 479001600      & 924            & 27720          \\\hline
15         & 1.176091   & 1.31e12        & 6435           & 360360         \\\hline
20         & 1.301030   & 2.43e18        & 184756         & 232792560      \\\hline
25         & 1.397940   & 1.55e25        & 5200300        & 2.68e10        \\\hline
30         & 1.477121   & 2.65e32        & 155117520      & 2.33e12        \\\hline
\end{tabular}
\end{table}


\subsection{因数,质数表}
\begin{table}[H]
\centering
\begin{tabular}{|l|llllll|}
\hline
\rowcolor[HTML]{C0C0C0} 
$n\le$              & \multicolumn{1}{l|}{\cellcolor[HTML]{C0C0C0}$10^1$}  & \multicolumn{1}{l|}{\cellcolor[HTML]{C0C0C0}$10^2$}  & \multicolumn{1}{l|}{\cellcolor[HTML]{C0C0C0}$10^3$}  & \multicolumn{1}{l|}{\cellcolor[HTML]{C0C0C0}$10^4$}    & \multicolumn{1}{l|}{\cellcolor[HTML]{C0C0C0}$10^5$}  & \cellcolor[HTML]{C0C0C0}$10^6$ \\ \hline
$\max\{\omega(n)\}$ & \multicolumn{1}{l|}{2}                               & \multicolumn{1}{l|}{3}                               & \multicolumn{1}{l|}{4}                               & \multicolumn{1}{l|}{5}                                 & \multicolumn{1}{l|}{26}                              & 7                              \\ \hline
$\max\{d(n)\}$      & \multicolumn{1}{l|}{4}                               & \multicolumn{1}{l|}{12}                              & \multicolumn{1}{l|}{32}                              & \multicolumn{1}{l|}{64}                                & \multicolumn{1}{l|}{128}                             & 240                            \\ \hline
$\pi(n)$            & \multicolumn{1}{l|}{4}                               & \multicolumn{1}{l|}{25}                              & \multicolumn{1}{l|}{168}                             & \multicolumn{1}{l|}{1229}                              & \multicolumn{1}{l|}{9592}                            & 78498                          \\ \hline
\rowcolor[HTML]{C0C0C0} 
$n\le$              & \multicolumn{1}{l|}{\cellcolor[HTML]{C0C0C0}$10^7$}  & \multicolumn{1}{l|}{\cellcolor[HTML]{C0C0C0}$10^8$}  & \multicolumn{1}{l|}{\cellcolor[HTML]{C0C0C0}$10^9$}  & \multicolumn{1}{l|}{\cellcolor[HTML]{C0C0C0}$10^{10}$} & \multicolumn{1}{l|}{\cellcolor[HTML]{C0C0C0}$10^{11}$} & $10^{12}$                        \\ \hline
$\max\{\omega(n)\}$ & \multicolumn{1}{l|}{8}                               & \multicolumn{1}{l|}{8}                               & \multicolumn{1}{l|}{9}                               & \multicolumn{1}{l|}{10}                                & \multicolumn{1}{l|}{10}                              & 11                             \\ \hline
$\max\{d(n)\}$      & \multicolumn{1}{l|}{448}                             & \multicolumn{1}{l|}{768}                             & \multicolumn{1}{l|}{1344}                            & \multicolumn{1}{l|}{2304}                              & \multicolumn{1}{l|}{4032}                            & 6720                           \\ \hline
$\pi(n)$            & \multicolumn{1}{l|}{664579}                          & \multicolumn{1}{l|}{5761455}                         & \multicolumn{1}{l|}{5.08e7}                          & \multicolumn{1}{l|}{4.55e8}                            & \multicolumn{1}{l|}{4.12e9}                          & 3.7e10                         \\ \hline
\rowcolor[HTML]{C0C0C0} 
$n\le$              & \multicolumn{1}{l|}{\cellcolor[HTML]{C0C0C0}$10^{13}$} & \multicolumn{1}{l|}{\cellcolor[HTML]{C0C0C0}$10^{14}$} & \multicolumn{1}{l|}{\cellcolor[HTML]{C0C0C0}$10^{15}$} & \multicolumn{1}{l|}{\cellcolor[HTML]{C0C0C0}$10^{16}$}   & \multicolumn{1}{l|}{\cellcolor[HTML]{C0C0C0}$10^{17}$} & $10^{18}$                        \\ \hline
$\max\{\omega(n)\}$ & \multicolumn{1}{l|}{12}                              & \multicolumn{1}{l|}{12}                              & \multicolumn{1}{l|}{13}                              & \multicolumn{1}{l|}{13}                                & \multicolumn{1}{l|}{14}                              & 15                             \\ \hline
$\max\{d(n)\}$      & \multicolumn{1}{l|}{10752}                           & \multicolumn{1}{l|}{17280}                           & \multicolumn{1}{l|}{26880}                           & \multicolumn{1}{l|}{41472}                             & \multicolumn{1}{l|}{64512}                           & 103680                         \\ \hline
$\pi(n)$            & \multicolumn{6}{l|}{$\pi(x)\sim x/\ln(x)$}                                                                                                                                                                                                                                                                          \\ \hline
\end{tabular}
\end{table}

\section{工具代码}
\subsection{快读}
\inputminted[breaklines, frame=single]{c++}
{../algo/杂项/快读.cpp}

\section{图论} % 一级标题

\subsection{Tarjan (强连通分量)} % 二级标题
\inputminted[breaklines, frame=single]{c++}
{../algo/图论/tarjan【强连通分量】.cpp}

\subsection{Tarjan (桥)} % 二级标题
\inputminted[breaklines, frame=single]{c++}
{../algo/图论/tarjan【桥】.cpp}


\subsection{欧拉路径 (无向图)} % 二级标题
\inputminted[breaklines, frame=single]{c++}
{../algo/图论/欧拉路径【无向图】.cpp}

\subsection{欧拉路径 (有向图)} % 二级标题
\inputminted[breaklines, frame=single]{c++}
{../algo/图论/欧拉路径【有向图】.cpp}



\subsection{最大流 (dinic)} % 二级标题
\inputminted[breaklines, frame=single]{c++}
{../algo/图论/网络流/dinic.cpp}


\subsection{费用流} % 二级标题
\inputminted[breaklines, frame=single]{c++}
{../algo/图论/网络流/最小费用最大流.cpp}

\section{数据结构}


\subsection{树状数组}
\inputminted[breaklines, frame=single]{c++}
{../algo/数据结构/树状数组.cpp}


\subsection{二维树状数组}
\inputminted[breaklines, frame=single]{c++}
{../algo/数据结构/二维树状数组.cpp}


\subsection{Segment Tree Beats(取min)}
\inputminted[breaklines, frame=single]{c++}
{../algo/数据结构/SGB【区间取min】.cpp}


\subsection{主席树}
\inputminted[breaklines, frame=single]{c++}
{../algo/数据结构/主席树.cpp}

\subsection{FHQ-Treap}
\inputminted[breaklines, frame=single]{c++}
{../algo/数据结构/权值无旋Treap.cpp}


\section{数学}
\subsection{逆元(线性递推)}
\inputminted[breaklines, frame=single]{c++}
{../algo/数学/逆元【O(n)预处理】.cpp}

\subsection{线性筛}
\inputminted[breaklines, frame=single]{c++}
{../algo/数学/欧拉筛.cpp}


\subsection{欧拉函数(单个)}
\inputminted[breaklines, frame=single]{c++}
{../algo/数学/欧拉函数【O(sqrt n)单个】.cpp}

\subsection{欧拉函数(线性递推)}
\inputminted[breaklines, frame=single]{c++}
{../algo/数学/欧拉函数【O(n)预处理】.cpp}

\subsection{拓展欧几里得算法(exgcd)}
\inputminted[breaklines, frame=single]{c++}
{../algo/数学/exgcd.cpp}


\subsection{中国剩余定理(CRT)}
\inputminted[breaklines, frame=single]{c++}
{../algo/数学/CRT.cpp}

\subsection{快速傅里叶变换(FFT)}
\inputminted[breaklines, frame=single]{c++}
{../algo/数学/FFT.cpp}

\subsection{快速数论变换(NTT)}
\inputminted[breaklines, frame=single]{c++}
{../algo/数学/NTT.cpp}

\section{字符串}

\subsection{KMP}
\inputminted[breaklines, frame=single]{c++}
{../algo/字符串/KMP.cpp}

\subsection{Z函数(拓展KMP)}
\inputminted[breaklines, frame=single]{c++}
{../algo/字符串/Z函数.cpp}


\subsection{AC自动机}
\inputminted[breaklines, frame=single]{c++}
{../algo/字符串/AC自动机.cpp}

\end{document}