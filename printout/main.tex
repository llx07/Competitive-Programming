\documentclass[twoside,twocolumn]{article}
\usepackage[UTF8]{ctex}
\usepackage{fancyhdr}  % 页眉页脚
\usepackage{minted}    % 代码高亮
\usepackage[colorlinks,linkcolor=black]{hyperref}  % 目录可跳转
\usepackage[table,xcdraw]{xcolor} % 表格背景颜色
\usepackage{amsmath} % operatorname 用
\usepackage{float}
\usepackage{geometry} % 调整页边距
\usepackage{enumitem} % 控制列表间距
\usepackage{graphicx}
\usepackage[fontsize=8pt]{fontsize}


% 调整页边距
\geometry{a4paper,scale=0.85}


% \setlength{\headheight}{5pt}

% 定义页眉页脚
\pagestyle{fancy}
\fancyhf{}
% \fancyhead[L]{\leftmark} % 当前section
\fancyhead[C]{ACM参考资料}
\fancyfoot[RO,LE]{\thepage}
% \setlength{\headheight}{5pt}

% 调整高亮风格
\usemintedstyle{default}

% 设置代码字体大小
\setminted{autogobble,breakanywhere}

\author{Lexiao Lin}   
\title{ACM参考资料}

\begin{document}
\maketitle % 封面
\newpage % 换页
\tableofcontents % 目录
\newpage

\section{基础知识}
\subsection{常用数学函数值}


\begin{table}[H]
\centering
\begin{tabular}{|l|l|l|l|l|}
\hline
\rowcolor[HTML]{C0C0C0} 
$n$ & $\log_{10} n$ & $n!$ & $C(n,n/2)$ & $\operatorname{lcm}(1...n)$ \\
2          & 0.301030   & 2              & 2              & 2              \\\hline
3          & 0.477121   & 6              & 3              & 6              \\\hline
4          & 0.602060   & 24             & 6              & 12             \\\hline
5          & 0.698970   & 120            & 10             & 60             \\\hline
6          & 0.778151   & 720            & 20             & 60             \\\hline
7          & 0.845098   & 5040           & 35             & 420            \\\hline
8          & 0.903090   & 40320          & 70             & 840            \\\hline
9          & 0.954243   & 362880         & 126            & 2520           \\\hline
10         & 1.000000   & 3628800        & 252            & 2520           \\\hline
11         & 1.041393   & 39916800       & 462            & 27720          \\\hline
12         & 1.079181   & 479001600      & 924            & 27720          \\\hline
15         & 1.176091   & 1.31e12        & 6435           & 360360         \\\hline
20         & 1.301030   & 2.43e18        & 184756         & 232792560      \\\hline
25         & 1.397940   & 1.55e25        & 5200300        & 2.68e10        \\\hline
30         & 1.477121   & 2.65e32        & 155117520      & 2.33e12        \\\hline
\end{tabular}
\end{table}




\subsection{因数,质数表}

\begin{table}[H]
\centering
\resizebox{\columnwidth}{!}{
\begin{tabular}{|l|llllll|}
\hline
\rowcolor[HTML]{C0C0C0} 
$n\le$              & \multicolumn{1}{l|}{\cellcolor[HTML]{C0C0C0}$10^1$}  & \multicolumn{1}{l|}{\cellcolor[HTML]{C0C0C0}$10^2$}  & \multicolumn{1}{l|}{\cellcolor[HTML]{C0C0C0}$10^3$}  & \multicolumn{1}{l|}{\cellcolor[HTML]{C0C0C0}$10^4$}    & \multicolumn{1}{l|}{\cellcolor[HTML]{C0C0C0}$10^5$}  & \cellcolor[HTML]{C0C0C0}$10^6$ \\ \hline
$\max\{\omega(n)\}$ & \multicolumn{1}{l|}{2}                               & \multicolumn{1}{l|}{3}                               & \multicolumn{1}{l|}{4}                               & \multicolumn{1}{l|}{5}                                 & \multicolumn{1}{l|}{26}                              & 7                              \\ \hline
$\max\{d(n)\}$      & \multicolumn{1}{l|}{4}                               & \multicolumn{1}{l|}{12}                              & \multicolumn{1}{l|}{32}                              & \multicolumn{1}{l|}{64}                                & \multicolumn{1}{l|}{128}                             & 240                            \\ \hline
$\pi(n)$            & \multicolumn{1}{l|}{4}                               & \multicolumn{1}{l|}{25}                              & \multicolumn{1}{l|}{168}                             & \multicolumn{1}{l|}{1229}                              & \multicolumn{1}{l|}{9592}                            & 78498                          \\ \hline
\rowcolor[HTML]{C0C0C0} 
$n\le$              & \multicolumn{1}{l|}{\cellcolor[HTML]{C0C0C0}$10^7$}  & \multicolumn{1}{l|}{\cellcolor[HTML]{C0C0C0}$10^8$}  & \multicolumn{1}{l|}{\cellcolor[HTML]{C0C0C0}$10^9$}  & \multicolumn{1}{l|}{\cellcolor[HTML]{C0C0C0}$10^{10}$} & \multicolumn{1}{l|}{\cellcolor[HTML]{C0C0C0}$10^{11}$} & $10^{12}$                        \\ \hline
$\max\{\omega(n)\}$ & \multicolumn{1}{l|}{8}                               & \multicolumn{1}{l|}{8}                               & \multicolumn{1}{l|}{9}                               & \multicolumn{1}{l|}{10}                                & \multicolumn{1}{l|}{10}                              & 11                             \\ \hline
$\max\{d(n)\}$      & \multicolumn{1}{l|}{448}                             & \multicolumn{1}{l|}{768}                             & \multicolumn{1}{l|}{1344}                            & \multicolumn{1}{l|}{2304}                              & \multicolumn{1}{l|}{4032}                            & 6720                           \\ \hline
$\pi(n)$            & \multicolumn{1}{l|}{664579}                          & \multicolumn{1}{l|}{5761455}                         & \multicolumn{1}{l|}{5.08e7}                          & \multicolumn{1}{l|}{4.55e8}                            & \multicolumn{1}{l|}{4.12e9}                          & 3.7e10                         \\ \hline
\rowcolor[HTML]{C0C0C0} 
$n\le$              & \multicolumn{1}{l|}{\cellcolor[HTML]{C0C0C0}$10^{13}$} & \multicolumn{1}{l|}{\cellcolor[HTML]{C0C0C0}$10^{14}$} & \multicolumn{1}{l|}{\cellcolor[HTML]{C0C0C0}$10^{15}$} & \multicolumn{1}{l|}{\cellcolor[HTML]{C0C0C0}$10^{16}$}   & \multicolumn{1}{l|}{\cellcolor[HTML]{C0C0C0}$10^{17}$} & $10^{18}$                        \\ \hline
$\max\{\omega(n)\}$ & \multicolumn{1}{l|}{12}                              & \multicolumn{1}{l|}{12}                              & \multicolumn{1}{l|}{13}                              & \multicolumn{1}{l|}{13}                                & \multicolumn{1}{l|}{14}                              & 15                             \\ \hline
$\max\{d(n)\}$      & \multicolumn{1}{l|}{10752}                           & \multicolumn{1}{l|}{17280}                           & \multicolumn{1}{l|}{26880}                           & \multicolumn{1}{l|}{41472}                             & \multicolumn{1}{l|}{64512}                           & 103680                         \\ \hline
$\pi(n)$            & \multicolumn{6}{l|}{$\pi(x)\sim x/\ln(x)$}                                                                                                                                                                                                                                                                          \\ \hline
\end{tabular}
}
\end{table}

\subsection{gprof 使用}
\begin{enumerate}
    \item 用gcc、g++、xlC编译程序时,使用-pg参数
    \item 执行编译后的可执行程序,生成文件gmon.out
    \item 使用gprof命令来分析记录程序运行信息的gmon.out文件(gprof test.exe gmon.out) 
\end{enumerate}

\newpage
\section{工具代码}
\subsection{快读}
\inputminted[breaklines, frame=single]{c++}{../algo/杂项/快读.cpp}


\subsection{快速IO}
\inputminted[breaklines, frame=single]{c++}{../algo/杂项/快速IO.cpp}


\subsection{无精度问题的向上向下取整除法}
\inputminted[breaklines, frame=single]{c++}{../algo/杂项/整数取整除法.cpp}


\section{图论} % 一级标题

\subsection{一般图}


% \subsubsection{最短路 (dijkstra)} % 二级标题
% \inputminted[breaklines, frame=single]{c++}{../algo/图论/dijkstra.cpp}


\subsubsection{Kruskal 重构树} % 二级标题


先构建 $n$ 个集合和 $n$ 个点权为 $0$ 的点,每个集合都有一个代表节点。

对于一颗最小生成树,按边权从小到大排序,对于每一条边 $(u,v,w)$,构建一个新的点权为 $w$ 的节点,作为集合 $u,v$ 代表节点的父亲,并将集合 $u,v$ 合并,代表节点更新为新节点。

这样我们就得到了一颗有 $2n-1$ 个节点,$n$ 个叶子的二叉树,这棵树就被称作 Kruskal 重构树。


在 Kruskal 重构树上,任意两个叶子节点的 LCA 的点权就是在原图上这两个节点之间所有路径最大边权的最小值。

由此我们看也可以得出,在原图上从节点 $u$ 出发经过所有边权 $\le w$ 的边能到达的节点,就是在 Kruskal 重构树上 $u$ 深度最小且点权 $\le w$ 的祖先 $v$ 的子树中所有叶子节点。

\subsubsection{Tarjan (强连通分量)} % 二级标题
\inputminted[breaklines, frame=single]{c++}{../algo/图论/tarjan【强连通分量】.cpp}

\subsubsection{Tarjan (桥)} % 二级标题
\inputminted[breaklines, frame=single]{c++}{../algo/图论/tarjan【桥】.cpp}

\subsubsection{Tarjan (割点)} % 二级标题
\inputminted[breaklines, frame=single]{c++}{../algo/图论/tarjan【割点】.cpp}

\subsubsection{欧拉路径 (有向图)} % 二级标题
\inputminted[breaklines, frame=single]{c++}{../algo/图论/欧拉路径【有向图】【不封装】.cpp}


\subsubsection{欧拉路径 (无向图)} % 二级标题
\inputminted[breaklines, frame=single]{c++}{../algo/图论/欧拉路径【无向图】【不封装】.cpp}


\subsubsection{匈牙利算法} % 二级标题
\inputminted[breaklines, frame=single]{c++}{../algo/图论/匈牙利算法.cpp}

\subsubsection{最大流 (dinic)} % 二级标题
\inputminted[breaklines, frame=single]{c++}{../algo/图论/网络流/dinic.cpp}


\subsubsection{费用流} % 二级标题
\inputminted[breaklines, frame=single]{c++}{../algo/图论/网络流/最小费用最大流.cpp}


\subsection{树}

\subsubsection{点分治} % 二级标题
\inputminted[breaklines, frame=single]{c++}{../algo/图论/树/点分治.cpp}

\subsubsection{DSU on tree} % 二级标题
\inputminted[breaklines, frame=single]{c++}{../algo/图论/树/dsu_on_tree.cpp}


\subsubsection{LCA (倍增)} % 二级标题
\inputminted[breaklines, frame=single]{c++}{../algo/图论/树/LCA【倍增】.cpp}


\subsubsection{LCA (欧拉序)} % 二级标题
记录欧拉序每个节点第一次出现的位置,两个位置之间深度最浅的节点就是 LCA。
\inputminted[breaklines, frame=single]{c++}{../algo/图论/树/LCA【欧拉序】.cpp}


\subsubsection{LCA (DFS 序)} % 二级标题
对于求出 $u,v$ 的 LCA 来说(不妨设 $\textit{dfn}_u\le \textit{dfn}_v$),$u=v$ 显然,否则在 DFS 序中 $[\textit{dfn}_u+1,\textit{dfn}_v]$ 区间最浅的节点的父亲就是 LCA。
\inputminted[breaklines, frame=single]{c++}{../algo/图论/树/LCA【dfs序】.cpp}



\subsubsection{树链剖分} % 二级标题
\inputminted[breaklines, frame=single]{c++}{../algo/图论/树/树链剖分【非封装版】.cpp}

\subsubsection{某些路径问题单log做法} % 二级标题


路径加/单点求和\&子树求和: 设 $a[u]$ 为子树内 $d[u]$ 之和,则

\begin{enumerate}
    \item 路径加:$d[u]+w$, $d[v]+w$, $d[lca(u,v)]-w d[fa[lca(u,v)]]-w$
    \item 单点求和:$a[u]=sumd(u)$
    \item 子树求和:$suma=\sum_v d[v]*(dep[v]-dep[u]+1)$,树状数组维护2系数即可
\end{enumerate}


单点加\&子树加/路径求和: 设 $g[u]$ 表示 $1$ 到 $u$ 权值之和,则

\begin{enumerate}
    \item 单点加:子树 $d$ 加
    \item 子树加 :$u$ 每个子树内 $v$ 有 $g[v] += w*(dep[v]-dep[u]+1)$ 则维护系数加即可
    \item 路径和:$sum(u,v)=g[u]+g[v]-g[lca(u,v)]-g[fa[lca(u,v)]]$
\end{enumerate}

\subsubsection{矩阵树定理} % 二级标题

Laplace 矩阵 $L=D-A$。


\begin{minipage}[t]{0.45\textwidth}
    $D_{i,i}$ 代表点的度数:
    \begin{itemize}[noitemsep, topsep=-5pt]
        \item 无向图:与 $i$ 相连边的权值和;
        \item 有向图叶向树:入边权值和;
        \item 有向图根向树:出边权值和;
    \end{itemize}
\end{minipage}
\hfill
\begin{minipage}[t]{0.45\textwidth}
    $A_{i,j}$ 表示 $i$ 和 $j$ 之间边数($i\ne j$):
    \begin{itemize}[noitemsep, topsep=-5pt]
        \item 无向图:$(i,j)$ 权值和;
        \item 有向图:$i\to j$ 权值和;
    \end{itemize}
\end{minipage}
\bigskip

而 $L$ 的 $(n-1)$ 主子式代表了生成树的个数:
\begin{itemize}[noitemsep, topsep=-5pt]
    \item 无向图:所有的 $(n-1)$ 主子式都相等,就表示生成树边权乘积和。
    \item 有向图:删去第 $k$ 行第 $k$ 列的主子式代表的就是以 $k$ 为根的叶向/根向树边权乘积和。
\end{itemize}


\section{数据结构}


\subsection{树状数组}
\inputminted[breaklines, frame=single]{c++}{../algo/数据结构/树状数组.cpp}


\subsection{二维树状数组}
\inputminted[breaklines, frame=single]{c++}{../algo/数据结构/二维树状数组.cpp}

% TODO 线段树二分

\subsection{01-Trie}
\inputminted[breaklines, frame=single]{c++}{../algo/数据结构/01-trie.cpp}

\subsection{Segment Tree Beats (取min)}
\inputminted[breaklines, frame=single]{c++}{../algo/数据结构/SGB【区间取min】.cpp}


\subsection{主席树}
\inputminted[breaklines, frame=single]{c++}{../algo/数据结构/主席树.cpp}

\subsection{FHQ-Treap}
\inputminted[breaklines, frame=single]{c++}{../algo/数据结构/权值无旋Treap(fhq).cpp}




\section{数学}
\subsection{逆元 (线性递推)}
\inputminted[breaklines, frame=single]{c++}{../algo/数学/逆元【O(n)预处理】.cpp}

\subsection{线性筛}
\inputminted[breaklines, frame=single]{c++}{../algo/数学/欧拉筛.cpp}


\subsection{欧拉函数 (单个)}
\inputminted[breaklines, frame=single]{c++}{../algo/数学/欧拉函数【O(sqrt n)单个】.cpp}

\subsection{欧拉函数 (线性递推)}
\inputminted[breaklines, frame=single]{c++}{../algo/数学/欧拉函数【O(n)预处理】.cpp}

\subsection{拓展欧几里得算法 (exgcd)}
\inputminted[breaklines, frame=single]{c++}{../algo/数学/exgcd.cpp}


\subsection{高斯消元}
\inputminted[breaklines, frame=single]{c++}{../algo/数学/高斯消元.cpp}

% TODO 高斯消元(异或)


\subsection{行列式 (取模)}
\inputminted[breaklines, frame=single]{c++}{../algo/数学/行列式【取模】.cpp}


\subsection{中国剩余定理 (CRT)}
\inputminted[breaklines, frame=single]{c++}{../algo/数学/CRT.cpp}

\subsection{快速傅里叶变换 (FFT)}
\inputminted[breaklines, frame=single]{c++}{../algo/数学/FFT.cpp}

\subsection{快速数论变换 (NTT)}
\inputminted[breaklines, frame=single]{c++}{../algo/数学/NTT.cpp}

\section{字符串}

\subsection{KMP}
\inputminted[breaklines, frame=single]{c++}{../algo/字符串/KMP.cpp}

\subsection{Z函数(拓展KMP)}
\inputminted[breaklines, frame=single]{c++}{../algo/字符串/Z函数.cpp}


\subsection{AC自动机}
\inputminted[breaklines, frame=single]{c++}{../algo/字符串/AC自动机.cpp}

% TODO AC自动机+拓扑排序优化

\subsection{后缀数组(SA)}
\inputminted[breaklines, frame=single]{c++}{../algo/字符串/SA.cpp}

\subsection{后缀自动机(SAM)}
\inputminted[breaklines, frame=single]{c++}{../algo/字符串/SAM.cpp}


\subsection{后缀自动机(SAM) map 版}
\inputminted[breaklines, frame=single, highlightlines={3,10,15,18}]{c++}{../algo/字符串/SAM(map版).cpp}

\subsection{回文树}
\inputminted[breaklines, frame=single]{c++}{../algo/字符串/PAM.cpp}


\subsection{回文划分 dp 优化}
\inputminted[breaklines, frame=single]{c++}{../algo/字符串/PAM-回文划分.cpp}

\subsection{Manacher}
\inputminted[breaklines, frame=single]{c++}{../algo/字符串/manacher.cpp}

\subsection{最小表示法}
\inputminted[breaklines, frame=single]{c++}{../algo/字符串/最小表示法.cpp}


% TODO 树哈希
% TODO 李超

% TODO 线性基

\end{document}